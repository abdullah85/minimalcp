%20171114-155627
\documentclass{article} 
\usepackage{multirow}
\usepackage{amsmath}
\usepackage{amssymb}
\usepackage{paralist}
\usepackage{algorithm, algorithmic}
\usepackage{mathtools}


%\definecolor{mygreen}{rgb}{0,0.6,0}
%\definecolor{mygray}{rgb}{0.5,0.5,0.5}
%\definecolor{mymauve}{rgb}{0.58,0,0.82}

%\newcommand{\A}{\mathcal A}
%\newcommand{\B}{\mathcal B}
%\newcommand{\C}{\mathcal C}
%\newcommand{\E}{\mathcal E}
%\newcommand{\agts}{\mathrm{\mathbf{Agt}}}
%\newcommand{\agts}{\mathit{Ag}}
%\newcommand{\act}{\mathit{Act}}
%\newcommand{\ha}{H_{\A}}
\newcommand{\deck}{\mathrm{Deck}}
%\newcommand{\szD}{|\bar s|}
%\newcommand{\deals}{\mathit{Deals(\bar s})}
%\newcommand{\Dl}{\mathit{Deal}}
%\newcommand{\dt}{\bar s}
%\newcommand{\citeA}{\cite{FG2016}}
%\newcommand{\rn}{\mathit{Runs}}
%\newcommand{\DS}{\operatorname{deal-safe}}
%\newcommand{\Safe}{\operatorname{(card-)safe}}
%\newcommand{\SSafe}{\operatorname{strongly (card-)safe}}
%\newcommand{\K}{\mathcal{K}}

%\newcommand{\z}{\text{Z3}}
%\newcommand{\cp}{\texttt{CP}}
%\newcommand{\solvr}{\texttt{Solver}}

%%%%%%%%%%%%%%%%%%%%%%%%%%%%%%%%%%%%%%%%%%%%%%%%%%%%%%%%%%%%%%%%



\begin{document}

\section{Announcement Types}

In what follows, we consider that agents
only make particular kinds of announcements.
We then consider protocols built up using such
announcements and perform an analysis of such
protocols. An agent $p$ may announce,
\begin{compactenum}
\item $ann(X)$ $\quad \coloneq \quad$ that his cards are a subset of $H_p \cup X$.
\item $\top$ $\quad \coloneq \quad$ an announcement equivalent to pass.
\end{compactenum}

\section{Protocols}

To construct a protocol, consider 
strategy \ref{s1} described below.

\begin{algorithm}
\floatname{algorithm}{Strategy}
\caption{S1($p$)}
\label{s1}
\begin{algorithmic}[1]
\STATE $ann(X_{p}^1)$ where $|X_p^1| = 2$ and $X_p^1 \cap H_p = \emptyset$ 
\STATE announce $ann(X_p^2)$ where $X_p^2 \subseteq X_P^1$
\end{algorithmic}
\end{algorithm}

A complete protocol $\pi_1$ is obtained when each agent
runs strategy \ref{s1} independently of each other. That
is, when making an announcement each agent is free to
choose $X_p^i$ as he wills (independent of history). 
Some points to note

\begin{compactenum}[\label=a)]
\item The announcement sequence is actually safe when others pass.
\item Nevertheless the final protocol might leak information to $\cal E$.
\item So, the question would be how bad is it? 
\item Does the protocol converge?
\item How much information is transferred among the agents?
\end{compactenum}

\section{Analysis}
In what follows, we tabulate the results for
sample executions for $\langle 4 | 4 | 4 \rangle$.
The setup as well as the executions used are detailed 
in Section \ref{appdx:exec}.

\subsection{Safety}
Safety depends on what Eaves knows (first order) after
any run. We tabulate some parameters relevant for
evaluating a run below. In particular, pos refers
to positive knowledge while neg refers to negative
knowledge and nD refers to number of deals.

\begin{table}[hc]
\centering
$ \qquad \qquad \qquad$
\\
\begin{tabular}{| c | c c c c c |}
 \hline
 & $r_0$ & $r_1$ & $r_2$ & $r_3$ & $r_4$ \\
 \hline
 pos & 8  & 8  & 9  & 8  & 10 \\
 neg & 20  & 18  & 21  & 20  & 22 \\
 nD & 3  & 7  & 3  & 3  & 2 \\
 \hline
\end{tabular}
\caption{Evaluating $\cal E$'s knowledge for each run}
\end{table}


\subsection{Informativity}
Informativity requirements are again influenced
by what any agent knows. We evaluate some relevant 
parameters (restricting to first order) and the results
are tabulated below. Note that for the same run each
agent may have a different perspective depending on the
actual deal.


\begin{table}[h]
$\quad$ % Table for a
\begin{tabular}{| c | c c c |}
 \hline
  & $d_0$ & $d_1$ & $d_2$ \\
 \hline 
 pos & 8  & 8  & 8 \\
 neg & 16  & 16  & 16 \\
 nD & 1  & 1  & 1 \\\hline 
\end{tabular}
$\quad$ % Table for b
\begin{tabular}{| c | c c c |}
 \hline
  & $d_0$ & $d_1$ & $d_2$ \\
 \hline 
 pos & 8  & 6  & 6 \\
 neg & 16  & 14  & 14 \\
 nD & 1  & 2  & 2 \\\hline 
\end{tabular}
$\quad$ % Table for c
\begin{tabular}{| c | c c c |}
 \hline
  & $d_0$ & $d_1$ & $d_2$ \\
 \hline 
 pos & 8  & 8  & 8 \\
 neg & 16  & 16  & 16 \\
 nD & 1  & 1  & 1 \\
\hline 
\end{tabular}
\caption{Run $r_0$}
\end{table}

\begin{table}[h]
$\quad$ % Table for a
\begin{tabular}{| c | c c c c c c c |}
 \hline
  & $d_0$ & $d_1$ & $d_2$ & $d_3$ & $d_4$ & $d_5$ & $d_6$ \\
 \hline
 pos & 8  & 8  & 8  & 8  & 6  & 6  & 8 \\
 neg & 16  & 16  & 16  & 16  & 14  & 14  & 16 \\
 nD & 1  & 1  & 1  & 1  & 2  & 2  & 1 \\\hline 
\end{tabular} 
\vspace{.25in}
$\quad$ % Table for b
\begin{tabular}{| c | c c c c c c c |}
 \hline
  & $d_0$ & $d_1$ & $d_2$ & $d_3$ & $d_4$ & $d_5$ & $d_6$ \\
 \hline
 pos & 5  & 5  & 6  & 6  & 6  & 6  & 5 \\
 neg & 13  & 13  & 14  & 14  & 14  & 14  & 13 \\
 nD & 3  & 3  & 2  & 2  & 2  & 2  & 3 \\\hline 
\end{tabular}
$\quad$ % Table for c
\begin{tabular}{| c | c c c c c c c |}
 \hline
  & $d_0$ & $d_1$ & $d_2$ & $d_3$ & $d_4$ & $d_5$ & $d_6$ \\
 \hline
 pos & 6  & 6  & 5  & 5  & 6  & 6  & 5 \\
 neg & 14  & 14  & 13  & 13  & 14  & 14  & 13 \\
 nD & 2  & 2  & 3  & 3  & 2  & 2  & 3 \\\hline 
\end{tabular}
$ \; \; \; $
\\
\caption{Run $r_1$}
\end{table}

\begin{table}[h]
$\quad$ % Table for a
\begin{tabular}{| c | c c c |}
 \hline
  & $d_0$ & $d_1$ & $d_2$ \\
 \hline
 pos & 8  & 8  & 8 \\
 neg & 16  & 16  & 16 \\
 nD & 1  & 1  & 1 \\\hline 
\end{tabular}
% Table for b
\hspace{17pt}
\begin{tabular}{| c | c c c |}
 \hline
  & $d_0$ & $d_1$ & $d_2$ \\
 \hline
 pos & 8  & 8  & 8 \\
 neg & 16  & 16  & 16 \\
 nD & 1  & 1  & 1 \\\hline 
\end{tabular}
\hspace{17pt}
% Table for c
\begin{tabular}{| c | c c c |}
 \hline
  & $d_0$ & $d_1$ & $d_2$ \\
 \hline
 pos & 5  & 5  & 5 \\
 neg & 13  & 13  & 13 \\
 nD & 3  & 3  & 3 \\\hline 
\end{tabular}
\caption{Run $r_2$}
\end{table}

\begin{table}[h]
$\quad$ % Table for a
\begin{tabular}{| c | c c c |}
 \hline
  & $d_0$ & $d_1$ & $d_2$ \\
 \hline
 pos & 8  & 8  & 8 \\
 neg & 16  & 16  & 16 \\
 nD & 1  & 1  & 1 \\\hline 
\end{tabular}
\hspace{13pt}
% Table for b
\begin{tabular}{| c | c c c |}
 \hline
  & $d_0$ & $d_1$ & $d_2$ \\
 \hline
 pos & 8  & 8  & 8 \\
 neg & 16  & 16  & 16 \\
 nD & 1  & 1  & 1 \\\hline 
\end{tabular}
\hspace{13pt}
% Table for c
\begin{tabular}{| c | c c c |}
 \hline
  & $d_0$ & $d_1$ & $d_2$ \\
 \hline
 pos & 6  & 6  & 8 \\
 neg & 14  & 14  & 16 \\
 nD & 2  & 2  & 1 \\\hline 
\end{tabular}
\caption{Run $r_3$}
\end{table}

\begin{table}[h]
$\qquad \qquad$ % Table for a
\begin{tabular}{| c | c c |}
 \hline
  & $d_0$ & $d_1$ \\
 \hline
 pos & 6  & 6 \\
 neg & 14  & 14 \\
 nD & 2  & 2 \\\hline 
\end{tabular}
\hspace{17pt}
% Table for b
\begin{tabular}{| c | c c |}
 \hline
  & $d_0$ & $d_1$ \\
 \hline
 pos & 8  & 8 \\
 neg & 16  & 16 \\
 nD & 1  & 1 \\\hline 
\end{tabular}
\hspace{17pt}
% Table for c
\begin{tabular}{| c | c c |}
 \hline
  & $d_0$ & $d_1$ \\
 \hline
 pos & 8  & 8 \\
 neg & 16  & 16 \\
 nD & 1  & 1 \\\hline 
\end{tabular}
\caption{Run $r_4$}
\end{table}


First of all, note that the deals that are listed
along the columns denote the set of deals which the
eavesdropper thinks possible for that run. This also
ensures that the run is a valid sequence of announcements
for that deal modulo first order reasoning. We have not
employed second order reasoning in any of the tables.

Further, note that the runs $r_0, r_2, r_3$ is converging for
agent $A$ no matter whatever deal we start in. Whereas for
run $r_2$, we note that the deals $d_4, d_5$ and are not
informative for run $r_1$. Similarly, note that runs 
$r_2, r_3, r_4$ are always converging for $B$ while 
runs $r_0, r_4$ are always converging for $C$.

Partial convergence is possible for $B$ for the
run $r_0$ while the run $r_3$ partially converges
for $C$. Using these we can come up with some suitable
measure to see how far each agent is from learning the
deal on the worst case as well as on an average.
For further detailed analysis of the runs,
we may look at the original runs 
listed in Section \ref{appdx:exec}.

\newpage
\appendix
% Code related to strategy 1.
\input{strategy1.tex}


\end{document}
%20171114-155627
