\section{Strategy 1}

\subsection{Encoding the strategy}

First the code for obtaining a valid
announcement sequence for an agent is given by
the function {\textbf \textrm getStrategy1}

\begin{verbatim}
def getStrategy1(deal, agt):
  '''
  Return a possible announcement sequence for agt at deal.
  Since announcements are independent of actual history, we
  can generate the strategy apriori.
  '''
  hand = deal[agt]
  rest = []
  for agt1 in deal.keys():
    if agt1 != agt:
      rest = rest + deal[agt1]  
  X = getCards(rest, 2)
  annL1 = ut.allHands(len(hand), hand + X)
  if rand.randint(0,1) == 0: 
    # drop an element of X
    X.pop()
  annL2 = ut.allHands(len(hand), hand + X)
  return [annL1, annL2]
\end{verbatim}

The code for {\textbf getCards} is as below,

\begin{verbatim}
def getCards(deck, n):
  '''
  Get n distinct cards randomly from deck
  '''
  if len(deck) < n:
    return []
  rangeList = []
  for c in deck:
    rangeList.append(c)
  cardsL = []
  for i in range(n):
    maxId = len(rangeList) - 1
    idx = rand.randint(0, maxId)
    cardsL.append( rangeList[idx])
    rangeList.pop(idx)
  return cardsL
\end{verbatim}

\subsection{Runs}

In order to obtain an actual run we use the code
{\textbf \textrm getRun1} shown below and 
{\textbf \textrm getRuns1} to get a list of 
distinct runs as given below.

\begin{verbatim}
def getRun1(deal, infAgts):
  '''
  Return a possible run of protocol 1 at deal.
  '''
  annSequences = {}
  for agt in infAgts:
    annSequences[agt] =  getStrategy1(deal, agt)
  run = []
  for i in range(2):
    for agt in infAgts:
      ann = (agt, annSequences[agt][i])
      run.append( ann )
  return run

def getRuns1(deal, infAgts, k, cutoff):
  '''
  Get k distinct runs of protocol 1 starting at deal.
  '''
  runList = []
  for i in range(k):
    currRun = getRun1(deal, infAgts)
    j = 0
    while currRun in runList and j < cutoff:
      currRun = getRun1(deal, infAgts)      
      j = j + 1
    if j == cutoff: # give up on obtaining k runs
      return runList
    runList.append(currRun)
  return runList
\end{verbatim}

\subsection{Analysis}
\label{apdx:s1Analysis}

What follows is the code related to analysis of
runs generated as per strategy $1$ detailed above.

\begin{verbatim}
def getEavesResult(state, run, eaves):
  '''
  Ensure that state is initialized to the required deal.
  Returns a record summarizing info that eaves learns after
  this run.
  '''
  result = {}
  s1 = state.execRun(run)
  result['pos'] = len(s1.getPosK(eaves))
  result['neg'] = len(s1.getNegK(eaves))
  result['nD']  = len(s1.getAgtDeals(eaves))
  return result
\end{verbatim}

\subsection{Experimental Results}
\label{appdx:exec}

Have a look at actually using above code to generate
some sample runs. The runs were generated in ipython
and the resulting runs were obtained below.


\begin{verbatim}
# IPython log file

import cpState as cps; import cpUtil  as ut ;
a,b,c,e = 'a', 'b', 'c', 'e'
dealLst = [4,4,4,0]; agtLst = [a, b, c, e];
d0 = ut.minDeal(dealLst, agtLst, range(12));
s0 = cps.cpState(dealLst, agtLst, d0, [a, b, c], e);

runList = getRuns1(d0, [a, b, c], 5, 3)

runList[0]
[('a',
  [[0, 1, 2, 3],  [0, 1, 2, 8],   [0, 1, 2, 10],
   [0, 1, 3, 8],  [0, 1, 3, 10],  [0, 1, 8, 10],
   [0, 2, 3, 8],  [0, 2, 3, 10],  [0, 2, 8, 10],
   [0, 3, 8, 10], [1, 2, 3, 8],   [1, 2, 3, 10],
   [1, 2, 8, 10], [1, 3, 8, 10],  [2, 3, 8, 10]]),
 ('b',
  [[1, 2, 4, 5],  [1, 2, 4, 6],   [1, 2, 4, 7],
   [1, 2, 5, 6],  [1, 2, 5, 7],  [1, 2, 6, 7],
   [1, 4, 5, 6],  [1, 4, 5, 7],  [1, 4, 6, 7],
   [1, 5, 6, 7], [2, 4, 5, 6],   [2, 4, 5, 7],
   [2, 4, 6, 7], [2, 5, 6, 7],  [4, 5, 6, 7]]),
 ('c',
  [[3, 4, 8, 9],  [3, 4, 8, 10],   [3, 4, 8, 11],
   [3, 4, 9, 10],  [3, 4, 9, 11],  [3, 4, 10, 11],
   [3, 8, 9, 10],  [3, 8, 9, 11],  [3, 8, 10, 11],
   [3, 9, 10, 11], [4, 8, 9, 10],   [4, 8, 9, 11],
   [4, 8, 10, 11], [4, 9, 10, 11],  [8, 9, 10, 11]]),
 ('a',
  [[0, 1, 2, 3],  [0, 1, 2, 8],   [0, 1, 2, 10],
   [0, 1, 3, 8],  [0, 1, 3, 10],  [0, 1, 8, 10],
   [0, 2, 3, 8],  [0, 2, 3, 10],  [0, 2, 8, 10],
   [0, 3, 8, 10], [1, 2, 3, 8],   [1, 2, 3, 10],
   [1, 2, 8, 10], [1, 3, 8, 10],  [2, 3, 8, 10]]),
 ('b', 
  [[1, 4, 5, 6], [1, 4, 5, 7], [1, 4, 6, 7], 
   [1, 5, 6, 7], [4, 5, 6, 7]]),
 ('c',
  [[4, 8, 9, 10],  [4, 8, 9, 11],  [4, 8, 10, 11],
   [4, 9, 10, 11], [8, 9, 10, 11]])]

runList[1]
[('a',
  [[0, 1, 2, 3],  [0, 1, 2, 7],   [0, 1, 2, 8],
   [0, 1, 3, 7],  [0, 1, 3, 8],  [0, 1, 7, 8],
   [0, 2, 3, 7],  [0, 2, 3, 8],  [0, 2, 7, 8],
   [0, 3, 7, 8], [1, 2, 3, 7],   [1, 2, 3, 8],
   [1, 2, 7, 8], [1, 3, 7, 8],  [2, 3, 7, 8]]),
 ('b',
  [[0, 4, 5, 6],  [0, 4, 5, 7],   [0, 4, 5, 8],
   [0, 4, 6, 7],  [0, 4, 6, 8],  [0, 4, 7, 8],
   [0, 5, 6, 7],  [0, 5, 6, 8],  [0, 5, 7, 8],
   [0, 6, 7, 8], [4, 5, 6, 7],   [4, 5, 6, 8],
   [4, 5, 7, 8], [4, 6, 7, 8],  [5, 6, 7, 8]]),
 ('c',
  [[0, 1, 8, 9],  [0, 1, 8, 10],   [0, 1, 8, 11],
   [0, 1, 9, 10],  [0, 1, 9, 11],  [0, 1, 10, 11],
   [0, 8, 9, 10],  [0, 8, 9, 11],  [0, 8, 10, 11],
   [0, 9, 10, 11], [1, 8, 9, 10],   [1, 8, 9, 11],
   [1, 8, 10, 11], [1, 9, 10, 11],  [8, 9, 10, 11]]),
 ('a',
  [[0, 1, 2, 3],  [0, 1, 2, 7],   [0, 1, 2, 8],
   [0, 1, 3, 7],  [0, 1, 3, 8],  [0, 1, 7, 8],
   [0, 2, 3, 7],  [0, 2, 3, 8],  [0, 2, 7, 8],
   [0, 3, 7, 8], [1, 2, 3, 7],   [1, 2, 3, 8],
   [1, 2, 7, 8], [1, 3, 7, 8],  [2, 3, 7, 8]]),
 ('b',
  [[0, 4, 5, 6],  [0, 4, 5, 7],   [0, 4, 5, 8],
   [0, 4, 6, 7],  [0, 4, 6, 8],  [0, 4, 7, 8],
   [0, 5, 6, 7],  [0, 5, 6, 8],  [0, 5, 7, 8],
   [0, 6, 7, 8], [4, 5, 6, 7],   [4, 5, 6, 8],
   [4, 5, 7, 8], [4, 6, 7, 8],  [5, 6, 7, 8]]),
 ('c',
  [[0, 1, 8, 9],  [0, 1, 8, 10],   [0, 1, 8, 11],
   [0, 1, 9, 10],  [0, 1, 9, 11],  [0, 1, 10, 11],
   [0, 8, 9, 10],  [0, 8, 9, 11],  [0, 8, 10, 11],
   [0, 9, 10, 11], [1, 8, 9, 10],   [1, 8, 9, 11],
   [1, 8, 10, 11], [1, 9, 10, 11],  [8, 9, 10, 11]])]

runList[2]
[('a',
  [[0, 1, 2, 3],  [0, 1, 2, 6],   [0, 1, 2, 7],
   [0, 1, 3, 6],  [0, 1, 3, 7],  [0, 1, 6, 7],
   [0, 2, 3, 6],  [0, 2, 3, 7],  [0, 2, 6, 7],
   [0, 3, 6, 7], [1, 2, 3, 6],   [1, 2, 3, 7],
   [1, 2, 6, 7], [1, 3, 6, 7],  [2, 3, 6, 7]]),
 ('b',
  [[1, 4, 5, 6],  [1, 4, 5, 7],   [1, 4, 5, 11],
   [1, 4, 6, 7],  [1, 4, 6, 11],  [1, 4, 7, 11],
   [1, 5, 6, 7],  [1, 5, 6, 11],  [1, 5, 7, 11],
   [1, 6, 7, 11], [4, 5, 6, 7],   [4, 5, 6, 11],
   [4, 5, 7, 11], [4, 6, 7, 11],  [5, 6, 7, 11]]),
 ('c',
  [[5, 6, 8, 9],  [5, 6, 8, 10],   [5, 6, 8, 11],
   [5, 6, 9, 10],  [5, 6, 9, 11],  [5, 6, 10, 11],
   [5, 8, 9, 10],  [5, 8, 9, 11],  [5, 8, 10, 11],
   [5, 9, 10, 11], [6, 8, 9, 10],   [6, 8, 9, 11],
   [6, 8, 10, 11], [6, 9, 10, 11],  [8, 9, 10, 11]]),
 ('a',
  [[0, 1, 2, 3],  [0, 1, 2, 6],   [0, 1, 2, 7],
   [0, 1, 3, 6],  [0, 1, 3, 7],  [0, 1, 6, 7],
   [0, 2, 3, 6],  [0, 2, 3, 7],  [0, 2, 6, 7],
   [0, 3, 6, 7], [1, 2, 3, 6],   [1, 2, 3, 7],
   [1, 2, 6, 7], [1, 3, 6, 7],  [2, 3, 6, 7]]),
 ('b', 
  [[1, 4, 5, 6], [1, 4, 5, 7], [1, 4, 6, 7], 
   [1, 5, 6, 7], [4, 5, 6, 7]]),
 ('c',
  [[5, 8, 9, 10], [5, 8, 9, 11], [5, 8, 10, 11],
   [5, 9, 10, 11],[8, 9, 10, 11]])]

runList[3]
[('a',
  [[0, 1, 2, 3],  [0, 1, 2, 6],   [0, 1, 2, 9],
   [0, 1, 3, 6],  [0, 1, 3, 9],  [0, 1, 6, 9],
   [0, 2, 3, 6],  [0, 2, 3, 9],  [0, 2, 6, 9],
   [0, 3, 6, 9], [1, 2, 3, 6],   [1, 2, 3, 9],
   [1, 2, 6, 9], [1, 3, 6, 9],  [2, 3, 6, 9]]),
 ('b',
  [[0, 4, 5, 6],  [0, 4, 5, 7],   [0, 4, 5, 10],
   [0, 4, 6, 7],  [0, 4, 6, 10],  [0, 4, 7, 10],
   [0, 5, 6, 7],  [0, 5, 6, 10],  [0, 5, 7, 10],
   [0, 6, 7, 10], [4, 5, 6, 7],   [4, 5, 6, 10],
   [4, 5, 7, 10], [4, 6, 7, 10],  [5, 6, 7, 10]]),
 ('c',
  [[4, 6, 8, 9],  [4, 6, 8, 10],   [4, 6, 8, 11],
   [4, 6, 9, 10],  [4, 6, 9, 11],  [4, 6, 10, 11],
   [4, 8, 9, 10],  [4, 8, 9, 11],  [4, 8, 10, 11],
   [4, 9, 10, 11], [6, 8, 9, 10],   [6, 8, 9, 11],
   [6, 8, 10, 11], [6, 9, 10, 11],  [8, 9, 10, 11]]),
 ('a',
  [[0, 1, 2, 3],  [0, 1, 2, 6],   [0, 1, 2, 9],
   [0, 1, 3, 6],  [0, 1, 3, 9],  [0, 1, 6, 9],
   [0, 2, 3, 6],  [0, 2, 3, 9],  [0, 2, 6, 9],
   [0, 3, 6, 9], [1, 2, 3, 6],   [1, 2, 3, 9],
   [1, 2, 6, 9], [1, 3, 6, 9],  [2, 3, 6, 9]]),
 ('b', 
  [[0, 4, 5, 6], [0, 4, 5, 7], [0, 4, 6, 7], 
   [0, 5, 6, 7], [4, 5, 6, 7]]),
 ('c',
  [[4, 8, 9, 10], [4, 8, 9, 11], [4, 8, 10, 11],
   [4, 9, 10, 11], [8, 9, 10, 11]])]

runList[4]
[('a',
  [[0, 1, 2, 3],  [0, 1, 2, 5],   [0, 1, 2, 7],
   [0, 1, 3, 5],  [0, 1, 3, 7],  [0, 1, 5, 7],
   [0, 2, 3, 5],  [0, 2, 3, 7],  [0, 2, 5, 7],
   [0, 3, 5, 7], [1, 2, 3, 5],   [1, 2, 3, 7],
   [1, 2, 5, 7], [1, 3, 5, 7],  [2, 3, 5, 7]]),
 ('b',
  [[3, 4, 5, 6],  [3, 4, 5, 7],   [3, 4, 5, 10],
   [3, 4, 6, 7],  [3, 4, 6, 10],  [3, 4, 7, 10],
   [3, 5, 6, 7],  [3, 5, 6, 10],  [3, 5, 7, 10],
   [3, 6, 7, 10], [4, 5, 6, 7],   [4, 5, 6, 10],
   [4, 5, 7, 10], [4, 6, 7, 10],  [5, 6, 7, 10]]),
 ('c',
  [[5, 7, 8, 9],  [5, 7, 8, 10],   [5, 7, 8, 11],
   [5, 7, 9, 10],  [5, 7, 9, 11],  [5, 7, 10, 11],
   [5, 8, 9, 10],  [5, 8, 9, 11],  [5, 8, 10, 11],
   [5, 9, 10, 11], [7, 8, 9, 10],   [7, 8, 9, 11],
   [7, 8, 10, 11], [7, 9, 10, 11],  [8, 9, 10, 11]]),
 ('a', [[0, 1, 2, 3], [0, 1, 2, 7], [0, 1, 3, 7], 
        [0, 2, 3, 7], [1, 2, 3, 7]]),
 ('b', [[4, 5, 6, 7], [4, 5, 6, 10], [4, 5, 7, 10], 
        [4, 6, 7, 10], [5, 6, 7, 10]]),
 ('c',
  [[5, 8, 9, 10],  [5, 8, 9, 11], [5, 8, 10, 11],
   [5, 9, 10, 11], [8, 9, 10, 11]])]

\end{verbatim}


