\section{Analysis}

\subsection{Strategy 1}
In this section, we tabulate some of the results of running the above 
strategies for various values of $k$ ranging from $2$ to $17$
as shown in Table \ref{leakageDTypes}. We observe that as
we increase $k$, the number of positive propositions leaked is
is close to $k$ whereas the number of negative propositions learnt
is about three to five times the value of $k$.

\begin{table}[h]
\begin{tabular}{| c || c | c | c | c | c |c | c | c | c | c | c | c | c | c | c | c |}
\hline
    $k$  &   2 &  3  &  4 & 5&  6 & 7 & 8 & 9 & 10 & 11 & 12 & 13 & 14 & 15 & 16 & 17 \\

\hline
% Actual data
  +ve    &   2 &  3  &  3 &  4 &  5 &  6 &  7 &  8 &  9   & 10  & 11 & 12 & 13 & 14 & 15 & 16\\
  -ve    &   8 &  12 &  15 &  19 &  23 &  27 &  31 &  35  & 39 & 43 & 48  & 51 & 55 & 59 &63 &67\\
  \#deals &   2 &  3  &  3 &  3 &  3 &  3 &  3 &  3 &  3  & 3  & 3 & 3 & 3 & 3 & 3 & 3\\
\hline
\end{tabular}
\caption{Information leakage across deal types}
\label{leakageDTypes}
\end{table}

%We now take a closer at a typical run of the protocol for the
%deal type $\langle 4, 4, 4 \rangle$
%\begin{verbatim}
%> ann1
%   ('a', [[0, 1, 2, 3], [0, 1, 2, 11], [0, 1, 3, 11], 
%          [0, 2, 3, 11], [1, 2, 3, 11]])
%> ann2 
%  ('c', [[1, 4, 5, 6], [3, 4, 5, 6], [8, 9, 10, 11]])
%
%>sf = s0.execRun([ann1, ann2])
%> sf.getAgtDeals(e)
%  [{'a':  [0, 1, 2, 3],  'b': [4, 5, 6, 7], 'c' : [8, 9, 10, 11], 'e': []},
%   {'a': [0, 2, 3, 11], 'b': [7, 8, 9, 10], 'c' : [1, 4,  5,  6], 'e': []},
%   {'a': [0, 1, 2, 11], 'b': [7, 8, 9, 10], 'c' : [3, 4,  5,  6], 'e': []}]
%
%> DL = sf.getAgtDeals(e)
%In [12]: for d in DL:
%    ...:     r0 = s0.newState()
%    ...:     r0.deal = d
%    ...:     r1 = r0.execRun([ann1, ann2])
%    ...:     print r1.getPosK(e)
%    ...:     
%    ...:     
%['Kea__0', 'Kea__2', 'Keb__7']
%['Kea__0', 'Kea__2', 'Keb__7']
%['Kea__0', 'Kea__2', 'Keb__7']
%\end{verbatim}
%
%Thus, no matter where we start, the set of positive propositions leaked
%remains the same. Similarly, we can also check for the number and kind
%of negative propositions leaked. We notice that there too there is no
%change across the possible deals.
%
%\begin{verbatim}
%In [13]: for d in DL:
%    ...:     r0 = s0.newState()
%    ...:     r0.deal = d
%    ...:     r1 = r0.execRun([ann1, ann2])
%    ...:     print r1.getNegK(e)
%    ...: 
%['KeNa__10', 'KeNa__4', 'KeNa__5', 'KeNa__6', 'KeNa__7', 
% 'KeNa__8', 'KeNa__9',  'KeNb__0', 'KeNb__1', 'KeNb__11', 
% 'KeNb__2', 'KeNb__3', 'KeNc__0', 'KeNc__2', 'KeNc__7']
%
%['KeNa__10', 'KeNa__4', 'KeNa__5', 'KeNa__6', 'KeNa__7', 
% 'KeNa__8', 'KeNa__9', 'KeNb__0', 'KeNb__1', 'KeNb__11', 
% 'KeNb__2', 'KeNb__3', 'KeNc__0', 'KeNc__2', 'KeNc__7']
%
%['KeNa__10', 'KeNa__4', 'KeNa__5', 'KeNa__6', 'KeNa__7', 
%  'KeNa__8', 'KeNa__9', 'KeNb__0', 'KeNb__1', 'KeNb__11', '
%  KeNb__2', 'KeNb__3', 'KeNc__0', 'KeNc__2', 'KeNc__7']
% \end{verbatim}

\subsection{Strategy 2}

For strategy 2, we ran some experiments for various deal
types of the form $\langle i,i,i,0\rangle$ and the results
are tabulated in Table \ref{leakageStrat2}. The fact that the
number of positive propositions leaked is $0$ conforms to our
assumption that it is safe.

\begin{table}[ht]
\begin{tabular}{| c || c | c | c | c | c |c | c | c | c | c | c | c | c | c | c | c |}
\hline
    $k$  &   2 &  3  &  4 & 5&  6 & 7 & 8 & 9 & 10 & 11 & 12 & 13 & 14 & 15 & 16 & 17 \\

\hline
% Actual data
  +ve    &   0 &  0 &  0 &  0 &  0 &  0 &  0 &  0 &  0 & 0  &  0 &  0 &  0 &  0 &  0 &  0 \\
  -ve    &   6 &  9 & 12 & 15 & 18 & 21 & 24 & 27 & 30 & 33 & 36 & 39 & 42 & 45 & 48 & 51 \\
  \#deals &  3 &  4 &  5 &  6 &  7 &  8 &  9 & 10 & 11 & 12 & 13 & 14 & 15 & 16 & 17 & 18 \\
\hline
\end{tabular}
\caption{Information leakage with Strategy 2}
\label{leakageStrat2}
\end{table}

\subsection{Three Steps and Beyond}

In what follows, we assume that the eavesdropper
has knowledge of the protocol. This in turn means
we can restrict our attention to protocols such that
each agent ``says what he means and means what he says''.

To further strengthen the protocol, we need
to go back to the first announcement. We note that
it is not strongly safe and hence we need to change it
if we need to strengthen it further. In order to maintain 
the information content transferred by Eave in the very first step
starting from the minimal deal $d_0$, we will have to choose a 
particular card (say $5$) and pair it with other cards to complete
a disjunct. If we have $D_1 $, $D_2$ in the
first announcement such that $D_1 \cap D_2 = \emptyset$, then,
there will be deals such that $B$ and $C$ will be uncertain. 

We set up the experiment as follows,
\begin{verbatim}
> dLst = ut.allDeals([4,4,4], [a,b,c], range(12))
> len(dLst)
34650
> cl  = ut.allHands(4, range(12))
> cl5 = filterHands(cL, [5])
> ann1L = [[0,1,2,3]] + cl5
> len(ann1L)
166
> ann1 = ('a', ann1L)
\end{verbatim}


As noted above, \text{ann1} is our required first announcement
that we are going to analyze with respect to second order reasoning
for the agents $B$ and $C$. It's clear that \text{ann1} is informative
to at least one of $B$ or $C$ as long as $A$ has the same hand as
he holds in the minimal deal. However, what happens when he holds other
hands that are a part of \text{ann1}?

\begin{verbatim}
> dL1 = filterDeals(dLst, a, cl5 + [[0,1,2,3]])
> len(dL1)
11620
\end{verbatim}

Basically, we first obtain the set of all deals where $A$
may announce $ann1$. There are exactly $70*(165+1) = 11620$ of
such deals and we obtained that set as shown above. Now,
we may analyze this set for leakage to each of the agents $B$
and $C$. The code for computing the leakage across deals is as
given below,

\begin{verbatim}
> def computeLeakage(s0, dLst, ann1, agt):
    kb = []
    for d in dLst: 
      r0 = s0.newState()
      r0.deal = d
      rF = r0.updateAnn(ann1)
      kb.append(rF.getCards(agt))
      kc.append(rF.getCards(c))
    return kb, kc
> %time kb.kc = computeLeakage(s0, dL1, ann1)

\end{verbatim}

% First Experiment
\begin{verbatim}
In [20]: fAnn
Out[20]: 
('a',
 [[0, 1, 2, 3],
  [0, 1, 2, 11],
  [0, 1, 3, 11],
  [0, 2, 3, 11],
  [1, 2, 3, 11],
  [0, 3, 4, 11],
  [2, 5, 6, 11],
  [4, 6, 7, 11]])
> kB = computeLeakage(s0, dL, fAnn, b) 
> kC = computeLeakage(s0, dL, fAnn, c)  
\end{verbatim}


% Note the error in running second leakage
% Second experiment 
\begin{verbatim}
In [31]: %time kB = computeLeakage(s0, dL, fAnn, b)
CPU times: user 47min 14s, sys: 3.3 s, total: 47min 17s
Wall time: 47min 24s

In [32]: %time kc = computeLeakage(s0, dL, fAnn, b)
CPU times: user 47min 21s, sys: 4.15 s, total: 47min 25s
Wall time: 50min 47s

In [33]: fAnn
Out[33]: 
('a',
 [[0, 1, 2, 3],
  [0, 1, 2, 7],
  [0, 1, 3, 7],
  [0, 2, 3, 7],
  [1, 2, 3, 7],
  [4, 6, 7, 9],
  [0, 2, 7, 9],
  [0, 3, 4, 7],
  [4, 5, 7, 11],
  [2, 7, 8, 11],
  [1, 4, 7, 10]])
\end{verbatim}
