\section{Direct Exchanges}

\subsection{Strategy 1}

Given the deal type $\langle k_1, k_2, k_3 \rangle$,
such that $k_1 = k_2 = k_3 = k$, do the following,

\begin{itemize}
  \item $A$ picks a card $x$ that he does not have and 
    announces $H_A \vee D_1 \vee \hdots D_n$ such that for 
    any $i$, (where $H_A = a_1 \hdots a_{k_1}$),
    \[ D_i = a_1 \wedge \hdots a_{i-1} \wedge  x \wedge a_{i+1} \hdots a_{k_1} \]
  \item $P \in \{B,C\}$, who has card $x$ responds with the announcement
    $H_P \vee D_1 \vee D_2$ where the $D_i$ are constructed as follows,
    \begin{itemize}
      \item Choose $2$ cards $a_1,a_2 \in H_A$ 
      \item Pick card $x_1 \in H_{Q}$ where $Q$ is neither $P$ nor $A$
      \item Let X be either $H_{Q} \backslash x_1$ or $H_P \backslash x$
      \item Finally, $D_1 = a_1 \wedge (H_Q \backslash x_1)$ and 
        $D_2 = a_2 \wedge (X)$
    \end{itemize}
  \item Note that both announcements are sorted before announcing.
\end{itemize}
The above strategy is implemented in python and the strategy is 
analysed in the next section. Here, we illustrate how we go about
setting up the system alongwith sample announcements generated.
\begin{verbatim}
> a,b,c,e = 'a', 'b', 'c', 'e'; 
> dL,agtL, cards = [4,4,4,0],[a,b,c,e], range(12); 
> deal = ut.minDeal(dL, agtL, cards); st = cps.cpState; 
> s0 = st(dL,agtL, deal, [a,b,c], e)

> ann1, x = firstAnn(s0, deal,a); ann2 = secondAnn(s0, deal, ann1) 
> ann1, ann2, x
 (('a', [[0, 1, 2, 5], [0, 1, 3, 5], [0, 2, 3, 5], [1, 2, 3, 5]]),
  ('b', [[1, 4, 6, 7], [2, 8, 9, 11], [4, 5, 6, 7]]),
  5)
\end{verbatim}

In the above run, $x$ was chosen as $5$, hence $P$ was $B$ due to $x$ and
finally $X$ was taken to be $H_P \backslash x$.

\subsection{Strategy 2}

Vary strategy 1 by modifying the second
announcement by allowing the number of disjuncts to be exactly 
equal to the number of disjuncts in first announcement.

Basically choose $D_i$ where $i \in [1,k]$ such that
\begin{compactenum}[a)]
\item $D_i = a_i \wedge X_i$ where 
  $X_i \subseteq (H_P \setminus x) \cup (H_Q)$
  and $|X_i| = (k-1)$.
\item For all $i$, $(X_i \cap H_{Q}) \neq \emptyset$ and $(\bigcap_i X_i) \cap H_P =  \emptyset$
\item Finally, $X_Q \subset (\cup_i X_i)  $.
%\item $(H_p\setminus x \cup H_Q) \subseteq (\bigcup_i X_i)$.
\end{compactenum}

The second player announces $(H_P \cup (\bigcup_i D_i))$.
To choose $X_i$, we have $\binom{2k-1}{k-1}$ possibilities
and it's clear that one can choose a collection of such
sets satisfying the conditions. In fact, one very straightforward
way is to ensure that $X_1 \subset H_Q$ in addition to
the second condition above. We claim that the above is a safe 
(even with second order reasoning) protocol.

%\subsection{Strategy 3}
%
%To further strengthen previous protocol, we see that it is not
%strongly safe. To make it strongly safe, clearly we need to change
%the first announcement. The following strategy is an attempt to 
%create a strongly safe protocol
%\begin{compactenum}[a)]
%\item $A$ announces $H_A \vee D_1 \vee D_2$ where 
%  \[D_1 \cup D_2 = (H_B \cup H_C) \text{ (entails } |D_1| = |D_2| = k \text{)}\].
%\item Now, $B$ announces $H_B \vee E_1 \vee E_2$ such that
%\begin{itemize}
%\item 
%\end{itemize}
%
%\end{compactenum}

\subsection{Three Steps and Beyond}

The advantage of the previous strategies was that the information
exchange was short and took place in exactly two meaningful announcements
(ignoring the announcements that said ``pass''). However, as we can observe
none of the runs are actually strongly safe and leaks information
to the Eavesdroppper. In order to rectify this, we need
to go back to the first announcement and analyse it more carefully
which we have done in the next section.
