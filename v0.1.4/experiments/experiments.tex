%20171114-155627
\documentclass{article} 
\usepackage{multirow}
\usepackage{amsmath}
\usepackage{amssymb}

\begin{document}

\section{Analysis}

Here, we analyze the three steps paper from 
\cite{ditmarsch:2011}. The cards leaked to the
agent $C$ are tabulated in table \ref{cLeaked-0}
while the deals that $C$ thinks possible after
each of those runs are tabulated in Table \ref{pDeals-0}.

With these tables, we are able to ascertain that 
our tool agrees with the analysis of \cite{ditmarsch:2011}
as it is a safe protocol and hence any run taken in
isolation is also guaranteed to be safe and hence $C$
does not learn any cards.

\begin{table}[h]
\begin{minipage}{0.5\linewidth}
\begin{tabular}{ | c || c | c | c | c | c | c  | } 
 \hline 
$nCards$  & $r_{0}$ & $r_{1}$ & $r_{2}$ & $r_{3}$ & $r_{4}$ & $r_{5}$ \\ 
 \hline 
0 & 5 & 5 & 5 & 5 & 5 & 5 \\ 
\hline
\end{tabular}
  \caption{Cards Leaked}
  \label{cLeaked-0}
\end{minipage}
\hfill \quad \begin{minipage}{0.5\linewidth}
  \begin{tabular}{ | c || c | c | c | c | c | c  | } 
    \hline$\;$   & $r_{0}$ & $r_{1}$ & $r_{2}$ & $r_{3}$ & $r_{4}$ & $r_{5}$ \\ 
    \hline 
    $pDeals$ & 5  & 5  & 5  & 5  & 5  & 5  \\ 
    \hline 
  \end{tabular}
  \caption{Possible Deals}
  \label{pDeals-0}
\end{minipage}
\hfill
%\caption{Cards Leaked and Possible Deals}
\end{table}

In the second table, we see that the number of possible
deals is also $5$ after every run and we can in fact prove
it independently that it would be exactly $5$ for each run. 
The above follows from the properties of the first announcement
made by $A$ that always consists of $15$ disjuncts of a
particular kind (type of $1$-designs and $2$-designs). 

The reasoning is that since the first announcement
is such that any card appears in exactly $6$ disjuncts and since
$C$ holds two cards, he can first eliminate $6$ hands of the announcement
for card $x$ that he holds. For the other card $y$, it occurs
in exactly $2$ hands alongwith $x$ and hence he can eliminate
$4$ other disjuncts leaving him with $5$ possibilities. The
later announcements do not reveal any additional information to
$C$ and hence he will be unsure of $5$ possible deals.

\section{Other Measures}

Besides the number of deals that the Eavesdropper thinks possible, we can 
consider other relevant measures of uncertainty after a particular run. 
These measures attempt to capture the remaining uncertainty as well
as vulnerabilities at a more abstract level (using atomic propositions).

For instance, we consider a proposition (or card) to be ``weak'' if
leaking the value of that proposition to the eavesdropper results
in him learning the complete deal. There are no propositions of that kind
for the protocol in \cite{ditmarsch:2011} (for the deal starting 
at $d_0 = <0,1,2,3 \;|\; 4,5,6,7 \;|\; 8,9>$ and run $r_0$).

\begin{verbatim}
> KL
[Kca__0, Kca__1, Kca__2, Kca__3, Kcb__4, Kcb__5, Kcb__6, Kcb__7]
> s1 = s0.execRun(r0)
> map(len, computeRemainingDeals(s1, KL))
  [3, 3, 3, 3, 3, 3, 3, 3]
\end{verbatim}

Thus even if we update the agent $C$'s knowledge with any
of the propositions, she still considers $3$ other possibilities
compatible with her knowledge.

We compute the same for other deals possible after $r_0$ and
we get similar property that $C$ is still unsure of the exact
deal for run $r_0$. However, note that the number of alternate
deals varies for different deals.

\begin{verbatim}
> alternateDeals = s1.getAgtDeals(c)
> for d in alternateDeals:
     s0.deal = d
     s1 = s0.execRun(r0)
     print map(len, computeRemainingDeals(s1, KL))

[3, 3, 3, 3, 3, 3, 3, 3]
[3, 3, 3, 3, 3, 3, 3, 3]
[3, 3, 3, 3, 3, 3, 3, 3]
[3, 3, 3, 3, 3, 3, 3, 3]
[3, 3, 3, 3, 3, 3, 3, 3]
\end{verbatim}

However, note that not all runs generate the same number of
deals when revealing a particular proposition. In fact,
if we take another run of the protocol $r_1$ and compute
the same measure above, we get that after revealing some
propositions, the number of remaining deals might be $2$
for the Eavesdropper as shown below.

\begin{verbatim}
> r1[0]
('a', [[2, 3, 7, 9],  [0, 1, 7, 9],  [2, 4, 6, 7],  
       [3, 5, 7, 8],  [0, 4, 5, 7],  [1, 6, 7, 8],  
       [2, 5, 8, 9],  [3, 4, 6, 9],  [0, 6, 8, 9],  
       [1, 4, 5, 9],  [0, 1, 2, 3],  [0, 2, 5, 6],
       [1, 2, 4, 8],  [0, 3, 4, 8],  [1, 3, 5, 6]])

> allDeals1, remDeals1, nRem1 = nDealsRemaining(r1, KL)
> nRem1
[[3, 2, 3, 2, 3, 2, 2, 3],
 [3, 2, 3, 2, 3, 2, 2, 3],
 [3, 2, 3, 2, 3, 2, 2, 3],
 [3, 2, 3, 2, 3, 2, 2, 3],
 [3, 2, 3, 2, 3, 2, 2, 3]]
\end{verbatim}

%> alternateDeals
%[{'a': [2, 3, 4, 5], 'b': [0, 1, 6, 7], 'c': [8, 9]},
% {'a': [0, 1, 2, 3], 'b': [4, 5, 6, 7], 'c': [8, 9]},
% {'a': [1, 3, 6, 7], 'b': [0, 2, 4, 5], 'c': [8, 9]},
% {'a': [0, 2, 6, 7], 'b': [1, 3, 4, 5], 'c': [8, 9]},
% {'a': [0, 1, 4, 5], 'b': [2, 3, 6, 7], 'c': [8, 9]}]

\subsection{Weak Pairs}

We can repeat the above experiment for weak pairs 
\begin{verbatim}
> pairKL
[[Kca__0, Kca__1], [Kca__0, Kca__2], [Kca__0, Kca__3], [Kca__1, Kca__3], 
 [Kca__2, Kca__3]]
\end{verbatim}

We get the following result (for $d_0$ and run $r_0$).
\begin{verbatim}
In [1]: %time allDeals0, pairRemDeals0, nPairRem0 = nDealsRemaining(r0, pairKL)
CPU times: user 36.2 s, sys: 44 ms, total: 36.3 s
Wall time: 36.4 s

In [238]: for l in nPairRem0:
     ...:     print l
     ...:     

[2, 2, 1, 1, 2, 2]
[2, 2, 1, 1, 2, 2]
[2, 2, 1, 1, 2, 2]
[2, 2, 1, 1, 2, 2]
[2, 2, 1, 1, 2, 2]

In [236]: %time allDeals0, pairRemDeals1, nPairRem1 = nDealsRemaining(r1, pairs)
CPU times: user 7.63 s, sys: 12 ms, total: 7.64 s
Wall time: 7.66 s

In [238]: for l in nPairRem1:
     ...:     print l
     ...:     
[1, 2, 1, 1, 2, 1]
[1, 2, 1, 1, 2, 1]
[1, 2, 1, 1, 2, 1]
[1, 2, 1, 1, 2, 1]
[1, 2, 1, 1, 2, 1]

\end{verbatim}


\bibliographystyle{plain}
%\bibliography{sadi}{}
\begin{thebibliography}{10}

\bibitem{ditmarsch:2011}
Hans~P. van Ditmarsch, and Fernando Soler-Toscano,
\newblock Three Steps.
\newblock In {\em Proceedings of the 12th International Conference on Computational Logic in Multi-agent Systems
{(CLIMA'11)}, Barcelona, Spain, 2011.}
%isbn = 978-3-642-22358-7 }
% pages = {41--57},
% numpages = {17},
% url = {http://dl.acm.org/citation.cfm?id=2044543.2044549},
% acmid = {2044549},
% publisher = {Springer-Verlag}, address = {Berlin, Heidelberg},

\end{thebibliography}

\newpage
\appendix
\section{Experimental Setup}

Note that the number of all hands that $A$ can have 
are $210$ and they can be ordered in ascending order
as shown in Table \ref{listHands}. The indices shown
in  that table are used in what follows. 

%\begin{table}

Table \ref{runsTab} shows a snapshot of the runs that were
used to check for leakage to agent $C$ as well as to compute the possible
deals that $C$ considers possible at the end of any run.
%Each possible hand of $A$ occurs in the first announcement of some run.

\begin{table}[h]
\begin{tabular}{| c | c | c | c | }
\hline
$\;$ & $0$ & $1$ & $2$ %& $3$ 
\\
\hline
\multirow{3}{*}{$0$}  
& 0, 26, 39, 52, 65, 
& 1, 11, 45, 58, 65,     
& 2, 7, 46, 58, 73, 
%& 3, 25, 29, 62, 66, 
%& 4, 16, 33, 50, 74, 79   
\\ 


& 82, 93, 110, 123, 136,  
& 79, 96, 113, 126, 136, 
& 76, 101, 119, 124, 133, 
%& 70, 94, 110, 119, 121, 
%& 79, 95, 115, 118, 123, 
\\

& 153, 159, 170, 181, 188 
& 143, 153, 166, 175, 204
& 140, 150, 163, 189, 196
%& 143, 169, 171, 179, 207
%& 147, 161, 162, 176, 209 
\\

\hline
\multirow{3}{*}{$\vdots$}  
& %0, 26, 39, 52, 65, 
& %1, 11, 45, 58, 65,     
& %2, 7, 46, 58, 73, 
%& 3, 25, 29, 62, 66, 
%& 4, 16, 33, 50, 74, 79   
\\ 


& $\hdots $ %82, 93, 110, 123, 136,  
& $\hdots$  %79, 96, 113, 126, 136, 
& $\hdots $ %76, 101, 119, 124, 133, 
%& 70, 94, 110, 119, 121, 
%& 79, 95, 115, 118, 123, 
\\

& %153, 159, 170, 181, 188 
& %143, 153, 166, 175, 204
& %140, 150, 163, 189, 196
%& 143, 169, 171, 179, 207
%& 147, 161, 162, 176, 209 
\\

\hline

%In [27]: i0
%Out[27]: [12, 13, 32, 36, 75, 
%          81, 96, 104, 114, 125, 
%         145, 161, 178, 182, 208]
%
%In [28]: i1
%Out[28]: [22, 27, 35, 41, 53, 
%          55, 92, 98, 108, 110, 
%         151, 152, 196, 199, 209]
%
%In [29]: i2
%Out[29]: [3, 13, 31, 57, 71, 
%         82, 104, 109, 114, 133, 
%         143, 155, 169, 186, 202]
%

\multirow{3}{*}{$11$}  
& 12, 13, 32, 36, 75,      
& 22, 27, 35, 41, 53, 
& 3, 13, 31, 57, 71, 
\\ 

& 81, 96, 104, 114, 125, 
& 55, 92, 98, 108, 110, 
& 82, 104, 109, 114, 133, 
\\

& 145, 161, 178, 182, 208
& 151, 152, 196, 199, 209
& 143, 155, 169, 186, 202
\\

\hline

\end{tabular}
\caption{First Announcements for $r_k$ (where $k = 3*i+j$)}
\label{runsTab}
\end{table}

%In [80]: i0
%Out[80]: [0, 26, 39, 52, 65, 
%          82, 93, 110, 123, 136, 
%          153, 159, 170, 181, 188]
%
%In [81]: i1
%Out[81]: [1, 11, 45, 58, 65, 
%          79, 96, 113, 126, 136, 
%          143, 153, 166, 175, 204]
%
%In [82]: i2
%Out[82]: [2, 7, 46, 58, 73, 
%          76, 101, 119, 124, 133, 
%          140, 150, 163, 189, 196]
%
%In [83]: i3
%Out[83]: [3, 25, 29, 62, 66, 
%          70, 94, 110, 119, 121, 
%          143, 169, 171, 179, 207]
%
%In [84]: i4
%Out[84]: [4, 16, 33, 50, 74, 
%          79, 95, 115, 118, 123, 
%          147, 161, 162, 176, 209]



Finally, the output obtained for card leakage for the $36$ runs 
as well as the actual deals that $C$ considers possible at the
end of each run (for any deal that can truthfully announce run)
are tabulated below.
%
\begin{minipage}\begin{tabular}{ | c || c | c | c | c | c | c  | } 
 \hline 
$nCards$  & $r_{0}$ & $r_{1}$ & $r_{2}$ & $r_{3}$ & $r_{4}$ & $r_{5}$ \\ 
 \hline 
0 & 5 & 5 & 5 & 5 & 5 & 5 \\ 
\hline
\end{tabular}
\end{minipage}\hfill


\begin{minipage}\begin{tabular}{ | c || c | c | c | c | c | c  | } 
 \hline 
$nCards$  & $r_{6}$ & $r_{7}$ & $r_{8}$ & $r_{9}$ & $r_{10}$ & $r_{11}$ \\ 
 \hline 
0 & 5 & 5 & 5 & 5 & 5 & 5 \\ 
\hline
\end{tabular}
\end{minipage}


\begin{minipage}\begin{tabular}{ | c || c | c | c | c | c | c  | } 
 \hline 
$nCards$  & $r_{12}$ & $r_{13}$ & $r_{14}$ & $r_{15}$ & $r_{16}$ & $r_{17}$ \\ 
 \hline 
0 & 5 & 5 & 5 & 5 & 5 & 5 \\ 
\hline
\end{tabular}
\end{minipage}


\begin{minipage}\begin{tabular}{ | c || c | c | c | c | c | c  | } 
 \hline 
$nCards$  & $r_{18}$ & $r_{19}$ & $r_{20}$ & $r_{21}$ & $r_{22}$ & $r_{23}$ \\ 
 \hline 
0 & 5 & 5 & 5 & 5 & 5 & 5 \\ 
\hline
\end{tabular}
\end{minipage}


\begin{minipage}\begin{tabular}{ | c || c | c | c | c | c | c  | } 
 \hline 
$nCards$  & $r_{24}$ & $r_{25}$ & $r_{26}$ & $r_{27}$ & $r_{28}$ & $r_{29}$ \\ 
 \hline 
0 & 5 & 5 & 5 & 5 & 5 & 5 \\ 
\hline
\end{tabular}
\end{minipage}


\begin{minipage}\begin{tabular}{ | c || c | c | c | c | c | c  | } 
 \hline 
$nCards$  & $r_{30}$ & $r_{31}$ & $r_{32}$ & $r_{33}$ & $r_{34}$ & $r_{35}$ \\ 
 \hline 
0 & 5 & 5 & 5 & 5 & 5 & 5 \\ 
\hline
\end{tabular}
\end{minipage}



\input{possTabs}
%\end{table}

\newpage
\begin{table}
\begin{verbatim}
> ut.allHands(4, range(10))
[[0, 1, 2, 3], [0, 1, 2, 4], [0, 1, 2, 5], [0, 1, 2, 6], [0, 1, 2, 7], 
 [0, 1, 2, 8], [0, 1, 2, 9], [0, 1, 3, 4], [0, 1, 3, 5], [0, 1, 3, 6], 
 [0, 1, 3, 7], [0, 1, 3, 8], [0, 1, 3, 9], [0, 1, 4, 5], [0, 1, 4, 6], 
 [0, 1, 4, 7], [0, 1, 4, 8], [0, 1, 4, 9], [0, 1, 5, 6], [0, 1, 5, 7], 
 [0, 1, 5, 8], [0, 1, 5, 9], [0, 1, 6, 7], [0, 1, 6, 8], [0, 1, 6, 9], 
 [0, 1, 7, 8], [0, 1, 7, 9], [0, 1, 8, 9], [0, 2, 3, 4], [0, 2, 3, 5], 
 [0, 2, 3, 6], [0, 2, 3, 7], [0, 2, 3, 8], [0, 2, 3, 9], [0, 2, 4, 5],
 [0, 2, 4, 6], [0, 2, 4, 7], [0, 2, 4, 8], [0, 2, 4, 9], [0, 2, 5, 6], 
 [0, 2, 5, 7], [0, 2, 5, 8], [0, 2, 5, 9], [0, 2, 6, 7], [0, 2, 6, 8], 
 [0, 2, 6, 9], [0, 2, 7, 8], [0, 2, 7, 9], [0, 2, 8, 9], [0, 3, 4, 5], 
 [0, 3, 4, 6], [0, 3, 4, 7], [0, 3, 4, 8], [0, 3, 4, 9], [0, 3, 5, 6], 
 [0, 3, 5, 7], [0, 3, 5, 8], [0, 3, 5, 9], [0, 3, 6, 7], [0, 3, 6, 8], 
 [0, 3, 6, 9], [0, 3, 7, 8], [0, 3, 7, 9], [0, 3, 8, 9], [0, 4, 5, 6], 
 [0, 4, 5, 7], [0, 4, 5, 8], [0, 4, 5, 9], [0, 4, 6, 7], [0, 4, 6, 8],
 [0, 4, 6, 9], [0, 4, 7, 8], [0, 4, 7, 9], [0, 4, 8, 9], [0, 5, 6, 7], 
 [0, 5, 6, 8], [0, 5, 6, 9], [0, 5, 7, 8], [0, 5, 7, 9], [0, 5, 8, 9], 
 [0, 6, 7, 8], [0, 6, 7, 9], [0, 6, 8, 9], [0, 7, 8, 9], [1, 2, 3, 4], 
 [1, 2, 3, 5], [1, 2, 3, 6], [1, 2, 3, 7], [1, 2, 3, 8], [1, 2, 3, 9], 
 [1, 2, 4, 5], [1, 2, 4, 6], [1, 2, 4, 7], [1, 2, 4, 8], [1, 2, 4, 9], 
 [1, 2, 5, 6], [1, 2, 5, 7], [1, 2, 5, 8], [1, 2, 5, 9], [1, 2, 6, 7], 
 [1, 2, 6, 8], [1, 2, 6, 9], [1, 2, 7, 8], [1, 2, 7, 9], [1, 2, 8, 9], 
 [1, 3, 4, 5], [1, 3, 4, 6], [1, 3, 4, 7], [1, 3, 4, 8], [1, 3, 4, 9], 
 [1, 3, 5, 6], [1, 3, 5, 7], [1, 3, 5, 8], [1, 3, 5, 9], [1, 3, 6, 7], 
 [1, 3, 6, 8], [1, 3, 6, 9], [1, 3, 7, 8], [1, 3, 7, 9], [1, 3, 8, 9], 
 [1, 4, 5, 6], [1, 4, 5, 7], [1, 4, 5, 8], [1, 4, 5, 9], [1, 4, 6, 7], 
 [1, 4, 6, 8], [1, 4, 6, 9], [1, 4, 7, 8], [1, 4, 7, 9], [1, 4, 8, 9], 
 [1, 5, 6, 7], [1, 5, 6, 8], [1, 5, 6, 9], [1, 5, 7, 8], [1, 5, 7, 9], 
 [1, 5, 8, 9], [1, 6, 7, 8], [1, 6, 7, 9], [1, 6, 8, 9], [1, 7, 8, 9], 
 [2, 3, 4, 5], [2, 3, 4, 6], [2, 3, 4, 7], [2, 3, 4, 8], [2, 3, 4, 9], 
 [2, 3, 5, 6], [2, 3, 5, 7], [2, 3, 5, 8], [2, 3, 5, 9], [2, 3, 6, 7], 
 [2, 3, 6, 8], [2, 3, 6, 9], [2, 3, 7, 8], [2, 3, 7, 9], [2, 3, 8, 9], 
 [2, 4, 5, 6], [2, 4, 5, 7], [2, 4, 5, 8], [2, 4, 5, 9], [2, 4, 6, 7], 
 [2, 4, 6, 8], [2, 4, 6, 9], [2, 4, 7, 8], [2, 4, 7, 9], [2, 4, 8, 9], 
 [2, 5, 6, 7], [2, 5, 6, 8], [2, 5, 6, 9], [2, 5, 7, 8], [2, 5, 7, 9], 
 [2, 5, 8, 9], [2, 6, 7, 8], [2, 6, 7, 9], [2, 6, 8, 9], [2, 7, 8, 9], 
 [3, 4, 5, 6], [3, 4, 5, 7], [3, 4, 5, 8], [3, 4, 5, 9], [3, 4, 6, 7], 
 [3, 4, 6, 8], [3, 4, 6, 9], [3, 4, 7, 8], [3, 4, 7, 9], [3, 4, 8, 9], 
 [3, 5, 6, 7], [3, 5, 6, 8], [3, 5, 6, 9], [3, 5, 7, 8], [3, 5, 7, 9], 
 [3, 5, 8, 9], [3, 6, 7, 8], [3, 6, 7, 9], [3, 6, 8, 9], [3, 7, 8, 9], 
 [4, 5, 6, 7], [4, 5, 6, 8], [4, 5, 6, 9], [4, 5, 7, 8], [4, 5, 7, 9], 
 [4, 5, 8, 9], [4, 6, 7, 8], [4, 6, 7, 9], [4, 6, 8, 9], [4, 7, 8, 9], 
 [5, 6, 7, 8], [5, 6, 7, 9], [5, 6, 8, 9], [5, 7, 8, 9], [6, 7, 8, 9]]
\end{verbatim}
\caption{All possible hands for A}
\label{listHands}
\end{table}

\end{document}
%20171114-155627

